\makeatletter
\def\input@path{{../}}
\makeatother
\documentclass[../main.tex]{subfiles}

\graphicspath{{ima/problemas}{../ima/problemas}}

% Aquí empieza el documento{{{
\begin{document}

\section{Problemas}%
\label{sec:problemas}

\thispagestyle{fancy}

\setcounter{subsection}{1}

\subsection{}%

En el circuito mostrado, obtén:

\begin{figure}[H]
	\centering
	\begin{tikzpicture}
		[
			circuit ee IEC,
			set resistor graphic=var resistor IEC graphic
		]
		\draw (0,0) to[resistor={info=$10\Omega$}] (5,0)
			node[contact] {} node[above right] {$c$}
			-- ++(0,2)
			to[resistor={info'=$4\Omega$}] ++(-2,0)
			node[contact] {} node[below] {$b$}
			to[battery={xscale=-1, info'=$5V$}] ++(-1,0)
			to[resistor={info'=$1\Omega$}] ++(-2,0)
			-- ++(0,-2)
			;
		\draw (5,2) -- ++(0,2)
			to[resistor={info'=$3\Omega$}] ++(-2,0)
			node[contact] {} node[below] {$a$}
			to[battery={xscale=-1, info'=$10V$}] ++(-1,0)
			to[resistor={info'=$2\Omega$}] ++(-2,0)
			-- ++(0,-2)
			;
	\end{tikzpicture}
\end{figure}


\begin{enumerate}[label=\alph*.]
	\item La corriente en los puntos $a$, $b$ y $c$.
	\item La diferencia de potencial $V_{ab}$ del punto $a$ respecto al
		punto $b$.
	\item El potencial en $a$ y $b$, si el punto $c$ es conectado a tierra.
\end{enumerate}

\setcounter{subsection}{4}

\subsection{}%

Para el circuito mostrado, las baterías no tienen resistencia interna
apreciable. Calcula:

\begin{figure}[H]
	\centering
	\begin{tikzpicture}
		[
			circuit ee IEC,
			set resistor graphic=var resistor IEC graphic
		]
		\draw (0,0) to[resistor={info'=$4\Omega$}, current direction={pos=0.9, info'=$3A$}]
			++(0,-2)
			-- ++(6,0)
			to[ current direction'={pos=0.1, info=$5A$}, resistor={info'=$6\Omega$} ]
			++(0,2)
			to[battery={pos=0.25, info'=$\mathcal{E}_1$},
			battery={pos=0.75, xscale=-1, info'=$\mathcal{E}_2$}] (0,0)
			;

		\draw (3,0) to[resistor={info=$3\Omega$}] ++(0,-2);

		\draw (0,0)
			-- ++(0,1)
			to[ current direction'={pos=0.25, info'=$2A$}, resistor={info=$R$}]
			++(6,0)
			--
			++(0,-1)
			;
	\end{tikzpicture}
\end{figure}

\begin{enumerate}[label=\alph*.]
	\item La resistencia $R$.
	\item La corriente en el resistor de $3\Omega$.
	\item Las $fem$ desconocidas $\mathcal{E}_1$ y $\mathcal{E}_2$.
\end{enumerate}

\setcounter{subsection}{10}

\subsection{}%

En la figura el amprerímetro $A_1$ tiene una resistencia interna de $2\Omega$,
mientras que el amprerímetro $A_2$ tiene una resistencia interna de $3\Omega$.
Si el condensador está completamente cargado determina:

\begin{figure}[H]
	\centering
	\begin{tikzpicture}
		[
			circuit ee IEC,
			set resistor graphic=var resistor IEC graphic
		]
		\draw (0,0) to[battery={info=$30V$, pos=0.25},
			resistor={info=$3\Omega$, pos=0.75, xscale=-1}]
			++(0,3)
			-- ++(3,0)
			to[capacitor={pos=0.25, info=$10\mu F$},
			resistor={pos=0.75, info'=$10\Omega$}]
			++(0,-3)
			-- (0,0);

		\draw (3,0) to[battery={pos=0.25, info'=$40V$},
			amperemeter={pos=0.5, info=$A_1$},
			resistor={pos=0.75, info=$4\Omega$},
			ground={pos=1}] ++(5,0);

		\draw (3,0)
			to[resistor={info'=$4\Omega$}, ground={pos=1}]
			++(45:3);

		\draw (1.5,3) to[
			amperemeter={pos=0.25, info=$A_2$},
			battery={pos=0.5, info=$20V$},
			resistor={pos=0.75, info=$4\Omega$},
			ground={pos=1}] ++(0,4);

	\end{tikzpicture}
\end{figure}

\begin{enumerate}[label=\alph*.]
	\item La lectura de cada amprerímetro.
	\item La carga almacenada en el capacitor.
\end{enumerate}
\end{document}
%}}}
