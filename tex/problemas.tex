\makeatletter
\def\input@path{{../}}
\makeatother
\documentclass[../main.tex]{subfiles}

\graphicspath{{ima/problemas}{../ima/problemas}}

% Aquí empieza el documento{{{
\begin{document}

\section{Problemas}%
\label{sec:problemas}

\thispagestyle{fancy}

\setcounter{subsection}{1}

\subsection{}%

En el circuito mostrado, obtén:

\begin{figure}[H]
	\centering
	\begin{tikzpicture}
		[
			circuit ee IEC,
			set resistor graphic=var resistor IEC graphic
		]

		\draw (0,0) to[resistor={info=$10\Omega$}] (5,0)
			node[contact] {} node[above right] {$c$}
			-- ++(0,2)
			to[resistor={info'=$4\Omega$}] ++(-2,0)
			node[contact] {} node[below] {$b$}
			to[battery={xscale=-1, info'=$5V$}] ++(-1,0)
			to[resistor={info'=$1\Omega$},
			current direction={pos=0.9, info=$I_3$}] ++(-2,0)
			to[current direction={pos=0.4, info'=$I_1$}]
			++(0,-2);

		\draw (5,2) -- ++(0,2)
			to[resistor={info'=$3\Omega$}] ++(-2,0)
			node[contact] {} node[below] {$a$}
			to[battery={xscale=-1, info'=$10V$}] ++(-1,0)
			to[resistor={info'=$2\Omega$}] ++(-2,0)
			to[current direction={pos=0.8, info'=$I_2$}]
			++(0,-2);

		\node at(1,1) {$M_1$};
		\node at(1,3) {$M_2$};

	\end{tikzpicture}
\end{figure}


\begin{enumerate}[label=\alph*.]
	\item La corriente en los puntos $a$, $b$ y $c$.
		\begin{align*}
			I_1 &= I_2 + I_3\\
			\intertext{$M_2$}
			\cancel{V_b} + 5 - 1I_3 + 2I_2 -10 + 3I_2 - 4I_3 &= \cancel{V_b}\\
			I_2 &= 1 + I_3\\
			\intertext{$M_1$}
			\cancel{V_c} - 4I_3 + 5 - 1I_3 - 10I_1 &= \cancel{V_c}\\
			I_3 + 2I_1 &= 1\\
			\\
			I_3 +2(I_2+I_3) &= 1\\
			I_2 &= \frac{1-3I_3}{2}\\
			1+ I_3 &= \frac{1-3I_3}{2}\\
			2 + 2I_3 &= 1 - 3I_3\\
			5I_3 &= -1\\
			\Aboxed
			{
				I_3 &= -\frac{1}{5}A = I_b
			}\\
			\\
			I_2 &= 1 + \left(- \frac{1}{5} \right)\\
			\Aboxed
			{
				I_2 &= \frac{4}{5}A = I_a
			}\\
			\\
			I_1 &= \left( \frac{4}{5} \right) + \left( - \frac{1}{5}  \right)\\
			\Aboxed
			{
				I_1 &= \frac{3}{5}A = I_c
			}
		\end{align*}
	\item La diferencia de potencial $V_{ab}$ del punto $a$ respecto al
		punto $b$.
		\begin{align*}
			V_b + 5 - 1I_3 + 2I_2 - 10 = V_a\\
			V_a - V_b &= -5 - \left( -\frac{1}{5} \right) +
			2 \left( \frac{4}{5} \right)\\
			V_a - V_b &= -5 + \frac{9}{5}\\
			V_a - V_b &= -\frac{16}{5}V
		\end{align*}
	\item El potencial en $a$ y $b$, si el punto $c$ es conectado a tierra.
		\begin{align*}
			0 - 4I_3 &= V_b\\
			V_b &= \frac{4}{5} V\\
			\\
			0 - 3I_2 &= V_a\\
			V_a &= - \frac{12}{5} V
		\end{align*}
\end{enumerate}

\setcounter{subsection}{4}

\subsection{}%

Para el circuito mostrado, las baterías no tienen resistencia interna
apreciable. Calcula:

\begin{figure}[H]
	\centering
	\begin{tikzpicture}
		[
			circuit ee IEC,
			set resistor graphic=var resistor IEC graphic
		]
		\draw (0,0) to[resistor={info'=$4\Omega$},
			current direction={pos=0.9, info'=$3A$}]
			++(0,-2)
			-- ++(6,0)
			to[ current direction'={pos=0.1, info=$5A$}, resistor={info'=$6\Omega$} ]
			++(0,2)
			to[
				current direction'={pos=0.1, info=$7A$},
				battery={pos=0.25, info'=$\mathcal{E}_1$},
				battery={pos=0.75, xscale=-1, info'=$\mathcal{E}_2$},
				current direction={pos=0.9, info'=$1A$}
			] (0,0)
			;

		\draw (3,0) to[resistor={info=$3\Omega$},
			current direction'={pos=0.85, info=$8A$}] ++(0,-2);

		\draw (0,0)
			-- ++(0,1)
			to[ current direction'={pos=0.25, info'=$2A$},
			resistor={info=$R$}]
			++(6,0)
			--
			++(0,-1)
			;

		\node at(1.5, -1) {$M_1$};
		\node at(4.5, -1) {$M_2$};
		\node at(3, 0.5) {$M_3$};
	\end{tikzpicture}
\end{figure}

\begin{enumerate}[label=\alph*.]
	\item La resistencia $R$.
		\begin{align*}
			\intertext{$M_{123}$}
			-2A*R - 3A*4\Omega + 5A*6\Omega &= 0\\
			2R &=  18\Omega\\
			\Aboxed
			{
				R &= 9\Omega
			}
		\end{align*}
	\item La corriente en el resistor de $3\Omega$.
		\begin{align*}
			3A+5A &= I_{3\Omega}\\
			\Aboxed
			{
				I_{3\Omega} &= 8A
			}
		\end{align*}
	\item Las $fem$ desconocidas $\mathcal{E}_1$ y $\mathcal{E}_2$.
		\begin{align*}
			\intertext{$M_1$}
			-8A*3\Omega + \mathcal{E}_2 - 3A*4\Omega &= 0\\
			\Aboxed
			{
				\mathcal{E}_2 &= 36V
			}
			\intertext{$M_2$}
			-8A*3\Omega + \mathcal{E}_1 - 5A*6\Omega &= 0\\
			\Aboxed
			{
				\mathcal{E}_1 &= 54V
			}
		\end{align*}
\end{enumerate}

\setcounter{subsection}{10}

\subsection{}%

En la figura el amprerímetro $A_1$ tiene una resistencia interna de $2\Omega$,
mientras que el amprerímetro $A_2$ tiene una resistencia interna de $3\Omega$.
Si el condensador está completamente cargado determina:

\begin{figure}[H]
	\centering
	\begin{tikzpicture}
		[
			circuit ee IEC,
			set resistor graphic=var resistor IEC graphic
		]
		\draw (0,0) to[current direction={pos=0.1, info=$I_1$},
			battery={info=$30V$, pos=0.25, xscale=-1},
			resistor={info=$3\Omega$, pos=0.75}]
			++(0,3)
			to[current direction={pos=0.6, info'=$A_3$}] ++(3,0)
			to[capacitor={pos=0.25, info=$10\mu F$},
			resistor={pos=0.75, info'=$10\Omega$}]
			++(0,-3)
			-- (0,0);

		\draw (1.5,3)
			to[
				current direction={pos=0.1, info=$I_2$},
				amperemeter={pos=0.25, info=$A_2$},
				battery={pos=0.5, info=$20V$, xscale=-1},
				resistor={pos=0.75, info=$4\Omega$},
				ground={pos=1}
			] ++(0,4);

		\draw (3,0)
			to[
				resistor={info'=$4\Omega$},
				ground={pos=1},
				current direction={pos=0.8, info'=$I_4$}
			]
			++(45:3);

		\draw (3,0)
			to[
				current direction'={pos=0.1, info=$I_5$},
				battery={pos=0.25, info'=$40V$},
				amperemeter={pos=0.5, info=$A_1$},
				resistor={pos=0.75, info=$4\Omega$},
				ground={pos=1}
			] ++(5,0);

	\end{tikzpicture}
\end{figure}

\begin{enumerate}[label=\alph*.]
	\item La lectura de cada amprerímetro.
	\item La carga almacenada en el capacitor.
\end{enumerate}
\end{document}
%}}}
