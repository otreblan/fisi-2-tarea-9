\makeatletter
\def\input@path{{../}}
\makeatother
\documentclass[../main.tex]{subfiles}

\graphicspath{{ima/problemas}{../ima/problemas}}

% Aquí empieza el documento{{{
\begin{document}

\section{Problemas}%
\label{sec:problemas}

\thispagestyle{fancy}

\setcounter{subsection}{1}

\subsection{}%

En el circuito mostrado, obtén:

\begin{enumerate}[label=\alph*.]
	\item La corriente en los puntos $a$, $b$ y $c$.
	\item La diferencia de potencial $V_{ab}$ del punto $a$ respecto al
		punto $b$.
	\item El potencial en $a$ y $b$, si el punto $c$ es conectado a tierra.
\end{enumerate}

\setcounter{subsection}{4}

\subsection{}%

Para el circuito mostrado, las baterías no tienen resistencia interna
apreciable. Calcula:

\begin{enumerate}[label=\alph*.]
	\item La resistencia $R$.
	\item La corriente en el resistor de $3\Omega$.
	\item Las $fem$ desconocidas $\mathcal{E}_1$ y $\mathcal{E}_2$.
\end{enumerate}

\setcounter{subsection}{10}

\subsection{}%

En la figura el amprerímetro $A_1$ tiene una resistencia interna de $2\Omega$,
mientras que el amprerímetro $A_2$ tiene una resistencia interna de $3\Omega$.
Si el condensador está completamente cargado determina:

\begin{enumerate}[label=\alph*.]
	\item La lectura de cada amprerímetro.
	\item La carga almacenada en el capacitor.
\end{enumerate}
\end{document}
%}}}
